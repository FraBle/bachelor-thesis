\section{Verwandte Forschungsarbeiten}\label{chap:relatedwork}

%%%%%%%%%%%%%%%%
%
%   Related Work
%
%%%%%%%%%%%%%%%%

Relevante Test-Daten sind in verschiedenen Schritten im Entwicklungsprozess eine Anwendung von Wichtigkeit.
Um Geschäftsanwendungen zu testen, gibt es neben dem Mittel die Datenbank-Anbindung zu mocken\footnote{\url{http://en.wikipedia.org/wiki/Mock_object}} auch die Möglichkeit sie mit einzubeziehen.
In diesem Kontext werden die Eingabewerte mit dem Zustand der Datenbank in Verbindung gebracht, wobei es dabei verschiedene Vorgehensweise gibt.

\subsection{Eingabewerte zum Testen von Datenbank-Anwendungen}
Mit dem Erzeugen von relevanter Eingabewerten für Anwendungstests haben sich in den letzten Jahren eine Reihe von Projekten beschäftigt.
Der Fokus der beschriebenen Ansätze liegt dabei, im Kontrast zu dieser Bachelorarbeit, auf der Code- und Branch-Abdeckung von Anwendungstests durch Einbeziehung des Status der Datenbank.

Im Sektor Datenbank-Anwendungs-Tests bietet das AGENDA Framework \cite{Chays:2000:FTD:347324.348954, Chays:2004:TDG:997669, Chays:2004:ATR:1077269.1077271, Deng:2005:TDT:1062455.1062486, Chays:2008:QTG:1385269.1385277} eine Palette an Tools für das funktionale Testen.
Dafür nutzt es Meta-Informationen aus der Datenbank in Kombination mit Voreinstellungen vom Entwickler um eine Test-Datenbank synthetisch zu erzeugen.

Auf Basis von Microsofts Testing-Framework Pex für die .Net-Plattform \cite{Tillmann:2008:PWB:1792786.1792798} entwickelten Pan et al. mehrere Erweiterungen \cite{Pan:2011:GPI:2190078.2190154, Pan:2011:DSG:1988842.1988846}, die die Code-Abdeckung durch Einbeziehung der Datenbank und ihrer Daten erhöhen.
Dabei werden mittels Dynamic Symbolic Execution (DSE) \cite{Cadar:2006:EAG:1180405.1180445, Godefroid:2005:DDA:1065010.1065036} die Variablen, die in SQL-Statements einfließen und ihren Änderungen nachverfolgt um daraus passende Programm-Eingaben zu generieren \cite{Pan:2011:GPI:2190078.2190154}.
Durch die zusätzlichen Anforderungen an Logical Coverage (LC) \cite{DBLP:conf/issre/AmmannOH03} und Boundary Value Coverage (BVC) \cite{DBLP:conf/issre/KosmatovLPU04} werden die gefunden Variablen zusätzlich noch im Zusammenhang mit Bedingungen innerhalb der Anwendung betrachtet, wodurch gegebenenfalls weitere Eingabe-Werte erzeugt werden. 

TODO: Weitere Paper betrachten!

\subsection{Live-Analyse von Datenbank-Anwendungen}
Eine alternative Möglichkeit die Performance eine Datenbank-Anwendung zu ermitteln, ist das Monitoring.
Software-Lösungen wie New Relic\footnote{\url{http://newrelic.com/}} betten eigene Komponenten in die laufende Anwendung ein um Metriken aus dem laufenden Betrieb aufzunehmen und zu analysieren.
Sie können dann unter anderem die langsamsten SQL-Statements mitsamt ihren Parameters dem Entwickler anzeigen.
Allerdings wird eine solche Analyse erst dann ausgeführt, wenn es schon zu spät sein kann: im produktiven Einsatz.
Die vorgestellte Entwicklungsumgebung soll hingegen Performance-Engpässe schon von vornherein aufdecken, sodass die Engpässe verhindert werden.
Eine Erweiterung der vorgestellten Algorithmen, Parameter aus den Ausführungsdaten zu extrahieren, würde beide Ideen kombinieren.

\subsection{Performance-Test-Frameworks}
