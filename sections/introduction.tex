\section{Einleitung}\label{chap:introduction}

%%%%%%%%%%%%%%%%
%
%    Einleitung
%
%%%%%%%%%%%%%%%%

In vielen Bereichen und vor allem bei Geschäftsanwendungen dienen relationale Datenbanken als essentielle Persistenz-Schicht.
Bei der Entwicklung solcher Anwendungen spielt neben der Validität auch die Performance eine wichtige Rolle.
Wichtigster Indikator dafür ist die tolerierbare Wartezeit, die sich laut psychologischen Studien schon nach 2 Sekunden negativ auf die Aufmerksamkeit der Nutzer auswirkt, das sie in ihrem Denkprozess unterbrochen werden \cite{Nah04}.
Ein zweiter wichtiger Aspekt ist die Entwicklung anhand von Echt-Daten.
Sie wird als beste Vorgehensweise betrachtet \cite[S. 212]{Plattner:2013:CID:2490529}, da sie die Charakteristiken der realen Welt als Maßstab nutzt.
Das frühzeitige Betrachten der Performance einer Anwendung auf Echt-Daten liegt so in der Verantwortung des Entwicklers und sollte von Anfang an in den Entwicklungsprozess einfließen.

Um Informationen aus der Datenbank in der Anwendung zu nutzen, erfolgt der Zugriff durch das Einbetten der, vorwiegend in SQL geschrieben, Anfragen.
Die eingebetteten Datenbankanfragen haben allerdings zumeist variable Bestandteile und Parameter, die für Performance-Messungen und -Analysen mit passenden Testwerten gefüllt werden müssen.
Häufig wird dies jedoch durch riesige Datenmengen in unverständlich benannten Tabellen und Attributen zusätzlich erschwert.
Um dies zu vereinfachen werden in dieser Bachelorarbeit verschiedene Ansätze diskutiert, die das Auswählen relevanter Testwerte durch sinnvolle Vorschläge anhand von Eigenschaften der Datenbank-Informationen unterstützen.

Die Ansätze bilden einen essentiellen Teil in einer Reihe von Konzepten für Entwicklungsumgebungen, die im \ref{chap:entwicklungsumgebung}. Kapitel zusammengetragen werden und bei der Entwicklung von Geschäftsanwendungen assistieren.
Anschließend wird die Einbettung von SQL in die Programmiersprache der Anwendungen untersucht um auf deren Basis die im Kapitel \ref{chap:testdatasuggestions} vorgestellten Algorithmen zur Vorschlagsgenerierung von Test-Daten zu erörtern.
Im Kapitel \ref{chap:testdataadministration} wird ergänzend die administrative Architektur für Test-Daten und -Systeme betrachtet.
Abschließend werden die vorgestellten Ansätze im Rahmen der Umsetzung einer realistischen Geschäftsanwendung evaluiert.