\section{Einleitung}\label{chap:introduction}

%%%%%%%%%%%%%%%%
%
%    Einleitung
%
%%%%%%%%%%%%%%%%

\nomenclature{IDE}{Integrated Development Environment}
\nomenclature{ERIC}{Enterprise-Ready IDE Concepts}
\nomenclature{SQL}{Structured Query Language}
\nomenclature{ER}{Entity-Relationship}
\nomenclature{JSON}{JavaScript Object Notation}
In vielen Bereichen, vor allem bei Geschäftsanwendungen, dienen relationale Datenbanken als essentielle Persistenzschicht für die anfallenden Daten.
Bei der Entwicklung solcher Anwendungen spielt neben der Validität auch die Performance eine wichtige Rolle.
Wichtigster Indikator dafür ist die tolerierbare Wartezeit, die sich laut psychologischen Studien schon nach 2 Sekunden negativ auf die Aufmerksamkeit der Nutzer auswirkt, sodass sie in ihrem Denkprozess unterbrochen werden \cite{Nah04}.
Ein zweiter wichtiger Aspekt ist die Entwicklung anhand von Echtdaten.
Sie wird als Best Practice betrachtet \cite[S. 212]{Plattner:2013:CID:2490529}, da sie die Charakteristiken der realen Welt als Maßstab nutzt.
Das frühzeitige Betrachten der Performance einer Anwendung auf Echtdaten liegt so in der Verantwortung des Entwicklers und soll von Anfang an in den Entwicklungsprozess einfließen.

Um Informationen aus der Datenbank in der Anwendung zu nutzen erfolgt der Zugriff durch das Einbetten von in SQL geschriebenen Anfragen.
Die eingebetteten Datenbankanfragen haben zumeist variable Bestandteile und Parameter, für die für Performance-Messungen und -Analysen passende Testwerte ausgewählt werden müssen.
Dies wird jedoch häufig durch riesige Datenmengen in unverständlich benannten Relationen und Attributen zusätzlich erschwert.
Mit dem Ziel der Vereinfachung werden in dieser Bachelorarbeit verschiedene Ansätze diskutiert, die das Auswählen relevanter Testwerte durch sinnvolle Vorschläge anhand der Daten und Metainformationen aus der Datenbank unterstützen.

Die vorgestellten Ansätze bilden einen essentiellen Teil in einer Reihe von Konzepten für Entwicklungsumgebungen, die im \ref{chap:entwicklungsumgebung}. Kapitel zusammengetragen werden und bei der Entwicklung von Geschäftsanwendungen assistieren.
Anschließend wird die Einbettung von SQL in die Programmiersprache der Anwendungen untersucht, um auf deren Basis die im Kapitel \ref{chap:testdatasuggestions} vorgestellten Algorithmen zur Vorschlagsgenerierung von Testdaten darzulegen.
Im Kapitel \ref{chap:testdataadministration} wird ergänzend die administrative Architektur für Testdaten und -systeme betrachtet.
Abschließend werden die vorgestellten Ansätze im Rahmen der Umsetzung einer praxisnahen Geschäftsanwendung evaluiert.