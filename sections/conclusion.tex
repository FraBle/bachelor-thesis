\section{Zusammenfassung und Ausblick}\label{chap:conclusion}

%%%%%%%%%%%%%%%%
%
%    Zusammenfassung und Ausblick
%
%%%%%%%%%%%%%%%%

TODO: Zusammenfassung und Ausblick


\subsection{Vorschläge auf Basis von Query-Plan-Analysen}
Um den Einfluss von bestimmten Parametern auf Abfrage-Ausführungsplan zu ermitteln, können die Bordmittel des Datenbanksystems als Ergänzung genutzt werden.
Die Analyse des SQL-Statements durch den SQL-Befehl \texttt{EXPLAIN PLAN} liefert dafür eine Kostenaufschlüsselung der einzelnen SQL-Operatoren vor der eigentlichen Ausführung.
Die Zuordnung von den Kosten zu den Parametern mit den zuvor ermittelten Testwerten würde eine Auskunft über deren Gewichtung geben.
Für eine Kostenanalyse inklusive Ausführung kann die SAP HANA interne Prozedur \texttt{PLANVIZ\_ACTION} genutzt werden.
Die Betrachtung einer solchen Analyse ist nicht Teil dieser Arbeit, kann aber in einer späteren Erweiterung die Präzision der Vorschläge von Testwerten erhöhen.

\subsection{Einbeziehung von Vorwissen über das genutzte System}
Sollte ein bestimmtes System genutzt werden, z.B. SAP ERP-Software, so kann Vorwissen über dessen Charakteristiken in den Vorschlägen zu Testdaten berücksichtigt werden.
Ein Beispiel dafür sind Standardwerte, die in jeder Instanz des Systems verwendet werden (z.B. feste Benutzerkennungen), oder die Betrachtung besonderer Zeiträume, wie das Jahresende.
Somit können Informationen aus dem Kontext des Systems die Vorschläge erweitern um typische Szenarien abzudecken.
