\clearpage
\section{SQL-Statements zur Bestimmung von Testwerten}

\begin{lstlisting}[caption={Bestimmung vorzuschlagender Testwerte anhand von Datencharakteristiken}, label={lst:distinctvalues}, language=mySQL, deletekeywords={schema, table, TABLE, SCHEMA, COLUMN, column}]
-- Häufigste Werte
SELECT <column>, COUNT(<column>) AS OCCURENCES
FROM <schema>.<table>
GROUP BY <column>
ORDER BY OCCURENCES DESC, <column> ASC
LIMIT 3;

-- Seltenste Werte
SELECT <column>, COUNT(<column>) AS OCCURENCES
FROM <schema>.<table>
GROUP BY <column>
ORDER BY OCCURENCES ASC, <column> ASC
LIMIT 3;

-- Bestimmung der Anzahl an distinkten Werten
SELECT COUNT(DISTINCT <column>) AS OCCURENCES
FROM <schema>.<table>;

-- OCCURENCES_HALF = OCCURENCES / 2

-- Werte um dem Median
SELECT <column>, COUNT(<column>) AS OCCURENCES
FROM <schema>.<table>
GROUP BY <column>
ORDER BY OCCURENCES ASC, <column> ASC
LIMIT 3 OFFSET OCCURENCES_HALF;
\end{lstlisting}
