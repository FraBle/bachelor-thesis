\section{BSEG- und BSIK-Erläuterung}

\begin{table}[h]
	\centering
	\begin{tabular}{ |l|l| }
		\hline
		\multicolumn{2}{|c|}{BSEG- bzw. BSIK-Spalten} \\
		\hline
		MANDT & Mandant \\
		GJAHR & Geschäftsjahr \\
		ZLSCH & Zahlweg \\
		BUKRS & Buchungskreis \\
		AUGDT & Datum des Ausgleichs \\
		LIFNR & Kontonummer des Lieferanten bzw. Kreditors \\
		SKFBT & Skontofähiger Betrag in Belegwährung \\
		WRBTR & Betrag in Belegwährung \\
		KUNNR & Debitorennummer \\
		BELNR & Belegnummer eines Buchhaltungsbeleges \\
		MANST & Mahnstufe \\
		\hline
	\end{tabular}
	\caption{Erklärung einer Auswahl an Spalten der BSEG- bzw. BSIK-Relation}
	\label{tab:bsegerlaeuterung}
\end{table}